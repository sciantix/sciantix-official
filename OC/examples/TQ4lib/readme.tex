\documentclass[12pt]{article}
\textwidth 165mm
\textheight 210mm
\oddsidemargin  1mm
\evensidemargin  1mm
\topmargin 1mm
\usepackage[latin1]{inputenc}

\begin{document}

\begin{center} 

{\Large \bf The Open Calphad Application Software Interface (OCASI)

Based on the TQ standard for interfacing thermodynamic software}

\bigskip

Bo Sundman \today

\end{center}

There is a Fortran version and a tentative iso-C version for C++.  
In the future it may be possible to merge these.

If you are not familiar with compiling and linking software and do not
understand the intructions here please ask someone close to you for
help.  The instructions here are very brief but I am too busy to
answer questions about handling such things and I know nothing about
C++

To link any of the examples you must first compile and link the OC
main program.  When this works you must compile a special library
excluding the file browser ``tinyfiledialogs'' and this is done (on
Windiws) by the command file {\bf makeocasilib}.  You must first add
the extension ``.cmd'' to this file and then execute it as a
command/batch file.  This generates the library files:

{\bf libocasi.a} and {\bf liboceqplus.mod}

Both of these files are needed to compile and link the applications.

The initial iso-C version of the library was provided by Teslos in
2014 and it has been extended by Matthias Stratmann at RUB, Germany
and Christophe Sigli at Constellium, France to handle more calls to
different subroutines.  As things are still under development there
may be slightly different versions on various subdirectories.

Files on this directory:

\begin{itemize}
\item readme.pdf is this file.  There are specific readme files on the
  subdirectories.

\item readme.tex is LaTeX source for this file.
  Subdirectories:
  \begin{itemize}
  \item F90 has the source code for the TQ library, liboctq.F90 that
    was updated 1019.10.31 (Halloween) and three subqdirectories with
    examples.
    \begin{itemize}
    \item The crfe/ was updated in October, 2019.
    \item feni/ has not been upd
      ated for a long time.
    \item parallel-alnipt/ simulating diffusion in Al-Ni-Pt in
      parallel added August 2021.  There are instructions how to use
      it in the directory.
    \end{itemize}

  \item Cpp has one C++ example provided by Matthias Stratmann at RUB,
    Germany and one from Cristophe Sigli, Consillium, France.  There
    is a separate version of the Fortran TQ library and an isoC
    interface.  I tested the Scheil program in February 2020 and it works
    but it generates some error messages I do not understand and as I
    do not know C++ I cannit fix that.  I would be greateful for any help.

    Note that STEP SCHEIL is now available as a command in OC.

  \end{itemize}
\end{itemize}

\end{document}
